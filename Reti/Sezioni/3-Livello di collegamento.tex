\documentclass{subfiles}
\begin{document}
    A livello di collegamento gli host e i router sono definiti come nodi mentre i canali di comunicazione che connettono i nodi sono 
    detti link e possono essere cablati (wired) o senza cavi (wireless). Il compito del data-link è quello di trasferire i datagram, 
    provenienti dal livello di rete, incapsulandoli in frame da un nodo ad un altro attraverso un link fisico.\\ \\
    Il data-link è implementato su tutti i nodi tramite una Network Interface Card (NIC), costituita da un controllore a una o più porte.
    Dal lato mittente. la NIC si occupa di incapsulare il datagram e segue il protocollo per accesso al canale e trasmissione. Dal lato 
    ricevitore, estrae il datagram e lo trasmette al livello di rete. Può essere dotato di sistema di rilevazione degli errori.\\ \\
    Oltre al trasporto dei datagram, i dettagli dei servizi possono variare in base al protocollo. I possibili servizi sono:
    \begin{itemize}
        \item Framing: incapsula i datagram in frame aggiungendo header e trailer;
        \item Accesso al collegamento: protocollo Medium Access Control (MAC) che regola l'accesso al collegamento e la trasmissione 
        dei frame tramite un indirizzo fisico contenuto nell'header;
        \item Consegna affidabile: i protocolli garantiscono il trasporto senza errori dei datagram. Viene realizzato tramite 
        acknowledgment, che conferma la ricezione dei frame, e la ritrasmissione, nel caso la consegna non avvenga come previsto;
        \item Controllo del flusso: protocolli che regolano la velocità di trasmissione tra i nodi;
        \item Rilevazione e correzione degli errori: gli errori possono essere generati da attenuazione e disturbi dei segnali. Si 
        possono individuare gli errori tramite tecniche su bit di controllo.
    \end{itemize}

    \subsection{Rilevazione e correzione degli errori}
    \subfile{Sotto/3-1-Rilevazione e correzione errori.tex}
    \clearpage

    

\end{document}