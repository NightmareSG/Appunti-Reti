\documentclass{subfiles}
\begin{document}
    Possiamo suddividere l'accesso alla rete in due tipologie: residenziale o aziendale, e chiaramente varia anche il tipo di tecnologia 
    utilizzata.
    \begin{Note*}
        Vi è un terzo tipo di accesso ovvero wireless su scala geografica con tecnologie e protocolli discussi successivamente.
    \end{Note*}

\subsubsection{Accesso residenziale}
    Per l'accesso residenziale a banda larga si hanno Digital Subscriber Line (DSL), via cavo e Fiber To The X (FTTX) .

    \paragraph{Accesso tramite DSL}
    La compagnia telefonica funge anche da ISP. Si usa come canale la linea telefonica (che risulterà poi come canale asimmetrico) per 
    scambiare i messaggi con un Digital Subscriber Line Access Multiplex (DSLAM) che si trova nella centrale locale. Il modem DSL 
    residenziale converte i dati digitali in toni ad alta frequenza per poterli trasmettere sul cavo telefonico ed una arrivata al DSLAM 
    essi vengono riconvertiti in segnali digitali. Si effettua multiplexing sulla banda di frequenza usata in modo da permettere l'uso 
    della linea telefonica e l'accesso ad Internet in contemporanea. Gli accessi residenziali DSL più comuni sono: ADSL e VDSL.

    \paragraph{Accesso via cavo}
    Nell'accesso ad Internet via cavo si utilizzano le infrastrutture esistenti della televisione. Tra i più comuni vi è la rete di 
    accesso ibrida con fibra ottica e cavo coassiale (HFC). Anche in questo caso vi è uno modem speciale detto cable modem che funziona 
    verosimilmente come il DSLAM. Inoltre si fa uso dello standard DOCSIS per la suddivisione della banda.

    \paragraph{FTTH}
    Nell'accesso FTTX la fibra arriva fino ad un determinato punto X della rete, in particolare per l'accesso residenziale si usa FTTH, 
    dove il collegamento in fibra ottica raggiunge una singola abitazione. La FTTH prevede che ogni abitazione abbia un Optical Network 
    Terminator (ONT) connesso ad uno splitter ottico per più abitazioni e ad un Optical Line Terminator (OLT) situato nella centralina. 
    L'OLT, tramite conversione da segnale ottico ad elettrico, si connette ad Internet tramite un router della compagnia, quindi 
    collegandosi ad esso si può accedere ad Internet.

\subsubsection{Accesso aziendale}
    Si utilizza una rete LAN per connettere i router di bordo. La tecnologia più usata è Ethernet. Tuttavia vi è anche una tipologia di 
    accesso a LAN wireless diffusa per i dispositivi mobili dove gli utenti possono inviare e ricevere pacchetti da e verso un access 
    point wireless entro un raggio di pochi metri.

    \begin{Note*}
        Le reti mobili sono di varie generazioni, dalla gestione di segnali analogici a digitali a cambi di frequenze e velocità di 
        trasmissione.
    \end{Note*}    

\end{document}