\documentclass{subfiles}
\begin{document}
    Per la rilevazione degli errori si utilizza uno schema Error Detection \& Correction (EDC), ovvero dei bit aggiunti di controllo. 
    Tuttavia potrebbero esservi degli errori non rilevati dal ricevente e per ridurre questa probabilità è necessaria un'elevata 
    ridondanza, ovvero calcoli più complessi e trasmissione di molti bit aggiuntivi. Vi sono tre tecniche per rilevare gli errori: 
    controllo di parità, checksum e controllo a ridondanza ciclica (CRC). 

\subsubsection{Controllo di parità}
    Vi sono due schemi che aggiungono un bit che assume un valore in modo tale da rendere pari, o dispari a secondo dello schema, i bit 
    a 1. Una volta trasmesso il frame si controlla il numero di bit a 1 per verificare se lo schema è stato rispettato o meno, ovvero si 
    verifica che non vi siano errori senza però sapere la sua posizione. Tuttavia se un errore cambia bit in numero pari, 
    indipendentemente dallo schema, al ricevente non appariranno errori trattando erroneamente il frame come corretto. \\ \\
    Si può optare per uno schema di parità bidimensionale, dove i bit dei dati vengono divisi in righe e colonne, ognuno con un proprio 
    bit di parità. In questo modo si può anche risalire al bit di errore.

\subsubsection{Checksum}
    Nella tecnica di Checksum si prende la sequenza a n bit del dato e la si divide in parti da 16 bit. Di queste sottosequenze effettua 
    la somma e ne effettua il complemento a 1 inviando il risultato. Il destinatario effettuerà lo stesso calcolo e sommerà il risultato 
    al complemento a 1, se il risultato è una sequenza di 1 allora non sono presenti errori, altrimenti viene scartato.

\subsubsection{Controllo a ridondanza ciclica}
    Questa tecnica prevede l'uso di codici di controllo a ridondanza ciclica detti anche codici polinomiali. Si prende il dato D e lo si 
    divide in insiemi da d bit, ognuno dei quali ha associato un polinomio b(x) di grado d-1. Mittente e destinatario si accordano su un 
    certo r che sarà il grado del polinomio generatore G, quindi il trasmittente divide b(x) per G e accoda il resto ottenuto al 
    messaggio inviandolo. Il destinatario conoscendo G si occuperà di dividere ciò che ha ricevuto per G, se il resto è 0 allora il 
    messaggio è integro, altrimenti presenta errori. Si noti che il CRC è in grado di individuare errori inferiori a r + 1 bit.
\end{document}