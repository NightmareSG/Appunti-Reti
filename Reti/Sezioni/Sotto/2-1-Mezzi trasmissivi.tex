\documentclass{subfiles}
\begin{document}
    I mezzi trasmissivi sono quelli che permettono il trasferimento e la propagazione dei bit da trasmettitore a ricevitore.\\
    Essi si dividono in:
    \begin{itemize}
        \item Vincolati: le onde vengono contenute in un mezzo solido come cavi di rame, fibra ottica ecc;
        \item Non vincolati: le onde si propagano nell'atmosfera e nello spazio esterno.
    \end{itemize}

\subsubsection{Mezzi trasmissivi vincolati}
    Essi trasportano segnali elettrici o luce e sono: doppini di rame, cavi coassiali e fibra ottica.

    \paragraph{Doppino di rame intrecciato(Twisted Pair)}
    Il doppino di rame è un cavo unidirezionale composto da due fili di rame e avvolti tra loro a spirale, questo perché intrecciandoli 
    i campi elettromagnetici si annullano a vicenda, e ogni coppia costituisce un collegamento di comunicazione.\\ \\
    Vengono divisi in categorie: fino alla categoria 6 vengono identificati come doppini non schermati(UTP) in quanto consistono solo in coppie e 
    un unico isolante che costituisce l'intero cavo, mentre i cavi di categoria 7 hanno un isolamento in più per ogni coppia in modo da 
    evitare eventuali interferenze dalle altre coppie.\\
    Sono spesso utilizzati per le reti LAN e possono raggiungere prestazioni elevate.

    \paragraph{Cavo coassiale}
    Il cavo coassiale è un cavo bidirezionale costituito da due conduttori di rame concentrici dove la parte superiore determina la 
    schermatura del cavo, in questa maniera si raggiungono alte frequenze di trasmissione. Sono utilizzati per reti ethernet o tv via 
    cavo e può essere utilizzato in condivisione tra sistemi periferici dove tutti ricevono quanto inviato da altri sistemi.

    \paragraph{Fibra ottica}
    La fibra ottica, anche detta Optical Carrier (OC), è composta da un tubo di vetro unidirezionale. Vi è un fotodiodo che converte 
    il segnale elettico in luminoso, esso attraversa il tubo di vetro e una volta arrivato all'altra estremità viene riconvertito 
    tramite un fotorecettore.

\subsubsection{Mezzi trasmissivi non vincolati}
    I mezzi trasmissivi non vincolati trasportano segnali nello spettro elettromagnetico, quindi non vi è un cavo, sono bidirezionali, 
    sono soggetti a ciò che riguarda l'ambiente di propagazione. Essi sono le onde radio nella banda terrestre e satellitare.

    \paragraph{Canali radio terrestri}
    I canali radio trasportano segnali tramite onde elettromagnetiche, inoltre dipendono dall'ambiente e quindi ostacoli presenti, 
    dalla distanza e sono soggetti a interferenze. Sono utilizzati per una LAN o un'area geografica.

    \paragraph{Canali radio satellitari}
    Si ha un satellite che per le comunicazioni collega delle ground station, ovvero trasmettitori a microonde terrestri. Si usano due 
    tipi di satelliti:
    \begin{itemize}
        \item Geostazionari: posizionati permanentemente ad una certa orbita dal suolo terrestre e sono sincronizzati alla rotazione 
        terrestre. La grande distanza provoca un ritardo nelle trasmissioni. Inoltre sono usati per le dorsali Internet;
        \item A bassa quota: posizionati più vicina orbitano intorno al pianeta. Possono comunicare anche tra loro oltre che con le 
        stazioni di terra. Inoltre per una buona copertura bisogna mandarene molti in orbita.
    \end{itemize}

    

\end{document}