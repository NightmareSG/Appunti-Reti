\documentclass{subfiles}
\begin{document}
    Un protocollo definisce il formato e l'ordine dei messaggi scambiati tra più entità in comunicazione, così come le azioni in fase 
    di trasmissione e/o ricezione.
    
\subsubsection{Progettazione}
    Nella progettazione di un protocollo si devono garantire affidabilità e evolvibility.

    \paragraph{Affidabilità}
    L'affidabilità di un protocollo è definità da:
    \begin{itemize}
        \item Capacità di ripristino dagli errori o guasti tramite error detection e error correction;
        \item Asseganzione delle risorse in maniera dinamica in base alle necessità tramite statistical multiplexing;
        \item Controllo del flusso affinché non vi sia un'inondazione di informazioni;
        \item Congestione, dovuta al traffico eccessivo di dati che la rete non riesce a gestire, non garantendo la consegna dei 
        pacchetti. 
    \end{itemize}

    \paragraph{Evolvibility}
    L'evolvibility consiste in:
    \begin{itemize}
        \item Implementazioni sia incrementali che scalari tramite protocol layering;
        \item identificazione di chi trasmette e riceve un messaggio. Questa procedura è detta Addressing nei livelli bassi e Naming 
        nei livelli alti.
    \end{itemize}

\subsubsection{Stratificazione}
    Una volta definito il sistema, si ricorre ad un modello di servizio per definire le operazioni possibili. Si individuano dei livelli, 
    ognuno dei quali implementa uno specifico servizio tramite operazioni che utilizzano servizi dello stesso livello o di un livello 
    inferiore. Ciò rende possibile identificare e relazionare elementi del sistema in modo da introdurre un modello di riferimento. 
    Questa procedura modulare facilita: progettazione, gestione e aggiornamento del sistema, implementazione di ulteriori servizi in 
    maniera trasparente a quelli già presenti.

\subsubsection{Architettura di rete}
    La maggior parte delle reti sono organizzate a livelli, ognuno dei quali offre servizi diversi ai livelli superiori. Ovviamente i 
    dati non sono trasferiti da un livello di una macchina allo stesso di un'altra in maniera diretta bensì passano attraverso i livelli 
    sottostanti fino al primo livello, il supporto fisico, che mette in atto la vera comunicazione e trasferimento.
    Tra ciascun livello vi è un'interfaccia che definsice le operazioni e servizi offerti dal livello inferiore a quello superiore.

    \begin{Note*}
        l'insieme dei protocolli usati da un sistema è detto protocol stack, che insieme ai livellli definisce la network architecture.
    \end{Note*}

    \paragraph{Impacchettamento dati}
    Come già detto i dati per essere trasmessi devono attraversare i livelli sottostanti. Durante questi passaggi vengono aggiunte 
    delle informazioni. Tra esse vi è l'header, aggiunto durante il primo attraversamento, che contiene un identificativo e
    informazioni per la consegna.
    \\ \\
    Possono essere imposti limiti dai protocolli come per la dimensione che porta a dover suddividere il messaggio in pacchetti dotati 
    di header e trailer. Una volta arrivati al destinatario il messaggio risale i vari livelli rimuovendo le informazioni aggiuntive e 
    consegnando il messaggio.    

\subsubsection{Connessione e affidabilità}
        Ogni livello offre a quello superiore due tipi di servizi e vari livelli di affidabilità. I servizi di connessione vengono 
        distinti tra orientati alla connessione e senza connessione.
    \begin{itemize}
        \item Orientato alla connessione: viene stabilita prima una connessione, durante la quale le due parti concordano i parametri da 
        usare(dimensione, qualità ecc), e una volta usata la si rilascia. I pacchetti arrivano a destinazione nell'ordine in cui sono 
        stati trasmessi;
        \item Senza connessione: anche detto datagram, ogni pacchetto ha l'indirizzo di destinazione ed è instradato in maniera 
        indipendente dagli altri pacchetti. Nel caso in cui il messaggio deve passare per altri nodi prima di arrivare a destinazione, 
        ogni nodo applicherà un approccio store \& forward.
    \end{itemize}
    Per quanto riguarda l'affidabilità di un servizio si intende la capacità di non perdere dati e consegnarli nell'ordine giusto. 
    Generealmente quindi si opta per un servizio orientato alla connessione con conferme di ricezione, che però possono appesantire 
    il sistema e causare ritardi. Alcune tecniche sono:
    \begin{itemize}
        \item message sequence: vengono rispettati i confini dei messaggi;
        \item byte stream: la connessione è un flusso di byte senza alcuna divisione.
    \end{itemize}

    \paragraph{Comunicazione}
    \begin{itemize}
        \item Comunicazione fisica: i dati seguendo un percorso fisico scendono la pila protocollare e dal mezzo fisico vengono 
        instradati al destinatario e risalgono fino al rispettivo livello di partenza;
        \item Comunicazione logica: ogni livello è distribuito in entità, ognuna delle quali implementano delle funzionalità, che 
        scambiano messaggi solo con i loro pari.
    \end{itemize}

\subsubsection{Primitive di servizio}
    Un servizio è formalmente descritto da un insieme di primitive, operazioni, che permettono di accedere al servizio. Questi set di 
    primitive differiscono in base al tipo di servizio, con o senza connessione.
    Tra quelle conosciute vi sono:
    \begin{itemize}
        \item LISTEN: attesa bloccante di una connessione in arrivo, indica che un server è pronto ad accettare una connessione;
        \item CONNECT: stabilisce una connessione con un peer in attesa, va specificato a chi connettersi, il client manda un pacchetto 
        e resta sospeso finché non riceve risposta;
        \item ACCEPT: accetta una richiesta di connessione da un peer, restituendo una risposta al client sbloccandolo;
        \item RECEIVE: attesa bloccante per un messaggio in arrivo, blocca il server in attesa di richieste;
        \item SEND: manda un messaggio dal peer al server e viceversa, sblocca il server per eseguire la richiesta o sblocca il client 
        per esaminare la risposta;
        \item DISCONNECT: termine della connessione, è una chiamata bloccante che sospende il client e spedisce un pacchetto al server 
        per rilasciare la connessione, il server esguirà una DISCONNECT per confermare l'operazione.
    \end{itemize}
    
\subsubsection{Protocolli non proprietari}
    L'impossibilità a comunicare tra reti diverse fece sviluppare due pile protocollari per creare uno standard di comunicazione. 
    Nacquero il TCP/IP e ISO/OSI.

    \paragraph{ISO/OSI}
    L'ISO/OSI è una pila protocollare strutturata in 7 livelli, dall'alto 3 orientati al contenuto dell'informazione e 4 orientati 
    alla rete. Si ha:
    \begin{itemize}
        \item Livello applicazione: si occupa della gestione della comunicazione delle applicazioni;
        \item Livello presentazione: si occupa della gestione della sintassi e codifica dei dati anche con codifiche diverse;
        \item Livello Sessione: si occupa di gestire le sessioni di utenti su macchine diverse, ovvero:
            \begin{itemize}
                \item organizza il dialogo tra due terminali;
                \item effettua il controllo del dialogo, uni o bidirezionale;
                \item gestisce la sicronizzazione.
            \end{itemize}
        \item Livello trasporto: si occupa di:
            \begin{itemize}
                \item trasferimento dati da sorgente a destinatario;
                \item gestione connessioni multiple;
                \item gestione canale punto-punto o senza garanzia dui dati;
                \item invio del messaggio a più destinazioni;
                \item frammentazione flusso in frame;
                \item correzione errori e prevenzione congestione.
            \end{itemize}
        \item Livello rete: si occupa di:
            \begin{itemize}
                \item instradare pacchetti in maniera dinamica o statica;
                \item congestione e qualità servizio.
            \end{itemize}
        \item Livello collegameto: datalink, si occupa di creare dei frame per la trasmissione tra elementi della rete locale e 
        e controllo degli errori e del flusso;
        \item Livello fisico: interfacce meccaniche ed elettriche per la trasmissione dei bit tramite canale di comunicazione.
    \end{itemize}
    Quindi questo modello è utilizzabile per la progettazione di un applicativo di rete.

    \paragraph{TCP/IP}
    Lo standard TCP/IP è diviso in 5 livelli:
    \begin{itemize}
        \item Applicazione: per i servizi;
        \item Trasporto: per il trasferimento dati host-host e può essere TCP, connection oriented, o UDP, connectionless;
        \item Rete o Internet: per l'instradamento dei datagram e protocolli di routing;
        \item Data Link o Collegamento: per il trasferimento dei frame lungo i mezzi di comunicazione;
        \item Fisico: trasferimento dei bit per mezzi fisici.
    \end{itemize}
    
    \begin{Note*}
        I livelli di presentazione e sessione del modello OSI vengono compensati dal sistema operativo del modello TCP.
    \end{Note*}
    
    \paragraph{Incapsulamento}
    L'incapsulamento è un processo attuato nel trasferimento dei dati. Dato il sistema di riferimento, ogni livello prende i dati dal 
    livello superiore, aggiunge un header per creare una nuova unità dati (PDU, Protocol Data Unit) e la passa al livello sottostante.
    
    \paragraph{Dati}
    I dati assumono nomi diversi in base al livello:
    \begin{itemize}
        \item Applicazione: messaggio o stream;
        \item Trasporto: segmenti o pacchetti;
        \item Rete: datagram;
        \item Collegamento: frame o trame.
    \end{itemize}
\end{document}