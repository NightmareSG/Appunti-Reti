\documentclass{subfiles}
\begin{document}
    Per inviare segnali digitali si devono comunque utilizzare segnali analogici che vanno modulati digitalmente. Le tecniche utilizzate 
    sono due:

    \paragraph*{Trasmissione in banda base}
    Consiste nell'uso di codifiche di linea come:
    \begin{itemize}
        \item NRZ (Non Return to Zero): poiché il segnale durante il traferimento può attenuarsi, il ricevente assegna i campioni ai 
        simboli più vicini: tensione positiva sarà 1, tensione negativa sarà 0. Si sfruttando tecniche di line code, come il clock, per 
        la migliore conversione dei bit appesantendo tuttavia la comunicazione;
        \item codifica Manchester: miscela il segnale di clock e i dati usando uno XOR, inviando un singolo segnale;
        \item NRZI (Non Return to Zero Inverted): si considera 1 come una transizione e lo 0 come situazione stazionaria;
        \item codifica bipolare o AMI: si usano quelli che vengono detti segnali bilanciati, ovvero dove si presentano tante tensioni 
        positive quanto negative su brevi periodi e quindi privi di componenti in corrente continua. Si usano due livelli di tensione 
        (opposti) per l'1, alternandoli in modo tale che in media si annullano, e 0 Volt per rappresentare lo 0.
    \end{itemize}

    \paragraph*{Trasmissione in banda passante}
    Una trasmissione che colloca un segnale confinato in una certa banda di frequenza arbitraria in modo da far coesistere diversi 
    segnali sullo stesso canale prende il nome di trasmissione in banda passante. In base a come si effettua la modulazione (ampiezze, 
    frequenze, fase) sulla banda passante si ottiene una modulazione digitale.

\subsubsection{Trasferimento dati}
    Esistono due approcci per il trasferimento dati:
    \begin{itemize}
        \item Commutazione di circuito (Circuit Switching): si necessita la creazione di una sessione end-to-end per il quale si crea un 
        circuito/canale costante. La banda viene suddivisa in più circuiti tramite Multiplexing che può avvenire per frequenza(FDM), 
        per frequenza dedicata, o tempo (TDM), per frame e slot;
        \item Commutazione di pacchetto (Packet Switching): ogni messaggio viene suddiviso in pacchetti che viaggiano nelle stesse 
        risorse di rete. Ciò può causare saturazione delle risorse, congestione del canale per le code dei pacchetti, ritardi dovuti 
        anche allo store \& forward, accodamento e disordine  nella ricezione dei pacchetti. 
    \end{itemize}
    Quindi si ha che la commutazione di pacchetto consente l'uso della rete a più utenti, mentre  per la commutazione di circuito si 
    deve tenere conto del numero massimo di utenti per canale e le probabilità di utenti attivi contemporaneamente.
   

\end{document}