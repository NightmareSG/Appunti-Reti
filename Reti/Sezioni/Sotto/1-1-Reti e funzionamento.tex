\documentclass{subfiles}
\begin{document}
    Internet può essere descritto come una rete di calcolatori interconnessi e composta da:
    \begin{itemize}
        \item Host, sistemi terminali, ovvero dispositivi della rete;
        \item Comunication Link, ovvero una rete di collegamento (costituita da fili di rame, fibre ottiche ecc) 
        dotata di velocità di trasmissione e lunghezza di banda;
        \item Commutatori di pacchetti, ovvero ulteriori dispositivi che inoltrano i pacchetti;
        \item ISP, Internet Service Provider, attraverso i quali si accede ad Internet;
        \item Protocolli di comunicazione, che controllano invio e ricezione dei messaggi.
    \end{itemize}
    
\subsubsection{Accesso alle informazioni}
    Esistono diversi modelli con funzioni diverse per lo scambio di informazioni.
    
    \paragraph{Modello Client-Server}
    Il modello Client-Server, che è alla base delle applicazioni network, è utilizzato 
    per l'accesso alle risorse dell'Internet.
    Vi sono:
    \begin{itemize}
        \item Client: colui che richiede le informazioni o qualsiasi servizio di rete;
        \item Server: colui che fornisce le informazioni o qualsiasi servizio di rete.
    \end{itemize}
    In esso sono coinvolti almeno due processi:
    \begin{itemize}
        \item il processo client che manda un messaggio, attraverso la rete, al processo server e resta in attesa di una risposta;
        \item il processo server che, una volta ricevuta una richiesta, esegue il lavoro/recupera dati e restituisce la risposta 
        al processo client.
    \end{itemize}

    \paragraph{Peer to Peer}
    Nel modello P2P ogni elaboratore può assumere il ruolo sia di client che di server. Data la multifunzionalità degli elaboratori, 
    essi hanno un proprio database delle informazioni quindi non sono necessari server dedicati. Utile al file sharing o telefonia 
    internet.

    \paragraph{Comunicazioni da persona a persona}
    Grazie al programma talk di UNIX si sono potute sviluppare applicazioni per la comunicazione tra persone come: instant message, 
    email, wiki, ecc.

\subsubsection{Tipologie delle reti}
    Le reti si possono distinguere per l'accesso a dei tipi di dati. Vi sono:
    \begin{itemize}
        \item Rete di accesso a banda larga e mobile;
        \item Reti di distribuzioni per contenuti o data center: esse muovono grandi quantità di dati tra i server tramite banda 
        trasversale ed inoltre sfruttano le CDN(Content Delivery Network) per connettere l'utente richiedente con il server più vicino;
        \item Reti di transito per l'accesso ai data center, può capitare che il servizio richiesto non risieda nella propria rete 
        quindi esso deve attraversare internet dal data center alla rete di accesso per poi finire nel dispositivo richiedente;
        \item Reti aziendali, che permette il file sharing tra dispositivi della rete.
    \end{itemize}
    \begin{Note*}
        i contenuti sono replicati in un singolo CDN e per decidere quale copia servire si tiene conto della distanza tra client e 
        replica, carico su ciascun server della CDN e traffico e congestione della rete.
    \end{Note*}

\subsubsection{Tecnologia di rete}
    Come già accennato, le reti possono differire per ampiezza del raggio d'azione. Abbiamo:

    \paragraph{Personal Area Network (PAN)}
    Le reti PAN connettono i dispositivi entro un raggio molto piccolo. Un esempio sono le reti che consentono la connessione di 
    periferiche al computer. Solitamente viene adottato il paradigma master-slave, dove il computer funge da master e comunica con le 
    periferiche con informazioni come indirizzi da usare, frequenza ecc.

    \paragraph{Local Area Network (LAN)}
    le reti LAN sono reti private in grado di operare all'interno o nelle vicinanze di un edificio. Consentono connettere tra loro gli 
    elaboratori tramite due tipi di connessione:
    \begin{itemize}
        \item LAN wireless: ogni PC è dotato di ricevitore radio e antenna e si connette agli altri tramite Access Point, router o 
        direttamente tramite configurazione mesh (reti simili a quelle wi fi di casa con punti di accesso simili);
        \item LAN cablata: usato diverse tecnologie di trasmissione. Sono estremamente veloce con una latenza molto bassa e quasi 
        assenza di errori di trasmissione. 
    \end{itemize}

    \paragraph{Metropolitan Area Network (MAN)}
    Le MAN sono reti che ricropono un'intera città. In esse i segnali vengono inseriti nella stazione di testa centralizzata per la 
    successiva distribuzione.

    \paragraph{Wide Area Network (WAN)}
    Le WAN ricoprono un'area geografica molto estesa, spesso una nazione o un continente. La rete è costituita da:
    \begin{itemize}
        \item Host e PC delle varie sedi;
        \item Subnet formata da linee di trasmissione e elementi di commutazione, generalmente router e/o switch. Quest'ultimi, 
        all'arrivo dei dati, devono scegliere la linea di uscita tramite forwarding e inoltrare i dati su di essa.
    \end{itemize}
    La maggior parte delle volte, le WAN sono formate da reti connesse da coppie di router. Per la comunicazione tra due si applicano 
    degli algoritmi di routing per trovare i percorsi e scegliere quello migliore.

    \paragraph{Internetwork}
    Date le differenze hardware/software tra le reti, si deve trovare un adattamento per poter effettuare la loro connessione, questo è 
    detto gateway. Quindi possiamo definire una internetwork come una rete di reti connesse tra di loro distinte e indipendenti.


\subsubsection{Architettura di Internet}
    Per collegarsi ad Internet, bisogna effettuare una conessione ad un ISP che fornisce diversi tipi di accesso (TV, fibre ottiche, DSL). 
    Una volta connessi all'ISP i nostri dati attraversano diversi ISP, tramite store \& forward, fino a quello del ricevente.\\
    Gli ISP sono dotati di un POP, Point of Presence, che permette appunto la connessione ad essi. Tale connessione può avvenire con:
    \begin{itemize}
        \item Peering Point: i router degli ISP sono connessi a coppie, utile per ISP molto estese;
        \item NAP: Neutral Access Point: i router hanno possono avere più di una connessione.
    \end{itemize}
    Gli ISP adottano una gerarchia a 3 livelli:
    \begin{itemize}
        \item ISP Tier 1: essi formano la dorsale di Internet, quindi tutti devono connettersi a loro per raggiungere il resto 
        dell'Internet, non pagano quindi il transito;
        \item ISP Tier 2: detti ISP regionali data la loro copertura, si connettono a Tier 1, pagando il transito, e altri Tier 2;
        \item ISP Tier 3: detti anche di accesso o last hop network. Sono i primi ISP a cui si effettua normalmente l'accesso.
    \end{itemize}
\end{document}